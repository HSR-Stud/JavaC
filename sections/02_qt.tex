\includepdf[pages=-]{Praktika/prak03/Loesung.pdf}

\section{Qt Manual Layout vs Layout Manager}
\begin{multicols}{2}
\textbf{Manuelles Layout} \\
int main(int argc, char *argv[])\{ \\
QApplication app(argc, argv); \\
app.setFont(QFont("Times", 15)); \\
QWidget wtop; \\
Ui-Objekt-Name * objname = \\
new Ui-Objekt-Name("TEXT"); \\
objname$->$setGeometry(x, y, w, h); \\
objname$->$setParent(\textbf{\&QWidget Name}); \\
\textit{\% Name des QWidgets: hier wtop}\\
objname$->$setAutoFillBackground(true); \\
objname$->$setStyleSheet("background-color: \textbf{color;}"); \\
wtop.show(); \\
return app.exec(); \}\\

Oft verwendete UI-Elemente: QLabel, QPushButton  \\
Bei QPushButton muss setAutoFillBackground und setStyleSheet nicht angewendet werden, da es schon automatisch geschieht.
\columnbreak \\
\textbf{Layout mit Layout Manager} \\
Layout-Typen: QHBoyLayout (horizontale Boxen), QVBoxLayout (verikale Boxen) \\
\textbf{Layout-Typ}* \textbf{layoutName} = \\
new \textbf{Layout-Typ}(\textbf{\&QWidget Name}); \\
Ui-Objekt-Name * objname = new Ui-Objekt-Name("TEXT"); \\
objname$->$setAutoFillBackground(true); \\
objname$->$setStyleSheet("background-color: \textbf{color;}"); \\
layoutName$->$addWidget(\textbf{name});

\textbf{Sub-Layout hinzufügen:} \\
Layout-Typ* sublayoutName = new Layout-Typ(); \\
layoutName$->$addLayout(sublayoutName); \\

\textbf{Unterschied zum Manuellen Layout:} \\
- Nur Hauptlayout muss mit QWidget bekannt gemacht werden. \\
- Sublayouts und Ui-Objekt wird via Hauptlayout mit QWidget-Objekt bekannt gemacht. \\
- Objektgeometrie muss nicht selbst definiert werden \\
\end{multicols}
\begin{minipage}[t]{0.5\textwidth}
\lstinputlisting{Praktika/prak03/Loesung/A2/SourceFiles/main.cpp}
\end{minipage}\hfill
\begin{minipage}[t]{0.48\textwidth}
\lstinputlisting{Praktika/prak03/Loesung/A3/SourceFiles/main.cpp}
\end{minipage}


\section{QObject auf Heap}
\begin{minipage}[t]{0.32\textwidth}
\subsection{main.cpp}
\lstinputlisting{Praktika/prak02/Loesungen/A3/src/main.cpp}

\end{minipage}
\begin{minipage}[t]{0.32\textwidth}
\subsection{Test.cpp}
\lstinputlisting{Praktika/prak02/Loesungen/A3/src/Test.cpp}

\end{minipage}
\begin{minipage}[t]{0.32\textwidth}
\subsection{Test.h}
\lstinputlisting{Praktika/prak02/Loesungen/A3/src/Test.h}

\end{minipage}

